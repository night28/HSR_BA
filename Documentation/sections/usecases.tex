\section{Use Cases} \label{usecases}

\subsection{Use Cases Brief}

\subsubsection{UC01: Network Orchestration}
Ein Administrator kann in einem Web Interface Netzwerke erstellen und verwalten. Des Weiteren ist die Möglichkeit gegeben, Policies, welche die Kommunikation zwischen diesen Netzen regeln zu erstellen. Die entsprechenden Konfigurationen werden auf allen beteiligten Geräten automatisch via APIs erstellt.

\subsubsection{UC02: ENCS Virtual Machine Management}
Ein Administrator kann in einem Web Interface die VMs auf einem ENCS System verwalten.

\subsubsection{UC03: Configuration History}
Ein Web Interface bietet die Möglichkeit, die Konfigurationen aller Geräte innerhalb einer Fabric anzuzeigen. Des weiteren kann die aktuelle Konfiguration mit älteren Versionen der Konfiguration verglichen werden.

\subsection{Use Cases Fully Dressed}

\subsubsection{UC01: Network Orchestration}
\begin{table}[H]
	\rowcolors{2}{gray!25}{white}
	\centering
	\begin{tabularx}{\textwidth}{l | X}
		Primary Actor      & Administrator        \\
		\hline
		Beschreibung       & Ein Administrator kann in einem Web Interface Netzwerke erstellen und verwalten. Des Weiteren ist die Möglichkeit gegeben, Policies, welche die Kommunikation zwischen diesen Netzen regeln zu erstellen. Die entsprechenden Konfigurationen werden auf allen beteiligten Geräten automatisch via APIs erstellt. \\ 
		\hline
		Stakeholders       &  
		\begin{itemize}	
			\item Administrator
			\item User
		\end{itemize}              \\
		\hline
		Preconditions      & 
		\begin{itemize}	
			\item DNA Center komplett konfiguriert
			\item Die Fabric läuft ohne Einschränkung
			\item Web Interface steht zur Verfügung
		\end{itemize}  \\
		\hline
		Postconditions     & 
		\begin{itemize}	
			\item Änderungen an Netzen wurden auf den Netzwerkdevices umgesetzt
			\item Policies sind korrekt umgesetzt
		\end{itemize}  \\
		\hline
		Main Success Story & 
		\begin{enumerate}
			\item Ein Netzwerk wird erstellt
			\item Die Software erstellt das Netzwerk auf allen nötigen Netzwerkgeräten
			\item Eine Policy für die Kommunikation mit anderen Netzen wird definiert
			\item Die Konfiguration für die Policy wird automatisch auf allen nötigen Netzwerkgeräten erstellt
		\end{enumerate}
		\\
		\hline
		Alternative Flows  & 
		\begin{itemize}
			\item[1a.] Ein Netzwerk wird gelöscht
			\item[1b.] Eine Policy wird verändert
		\end{itemize}
	\end{tabularx}
	\caption{UC01 Fully Dressed}
	\label{tab:UC01}
\end{table}

\subsubsection{UC02: ENCS Virtual Machine Management}
\begin{table}[H]
	\rowcolors{2}{gray!25}{white}
	\centering
	\begin{tabularx}{\textwidth}{l | X}
		Primary Actor      & Administrator        \\
		\hline
		Beschreibung       & Ein Administrator kann in einem Web Interface die VMs auf einem ENCS System verwalten. \\ 
		\hline
		Stakeholders       &  
		\begin{itemize}	
			\item Administrator
		\end{itemize}              \\
		\hline
		Preconditions      & 
		\begin{itemize}	
			\item ENCS ist konfiguriert und mit dem DNA Center verbunden
			\item VM Profile existieren
		\end{itemize}  \\
		\hline
		Postconditions     & 
		\begin{itemize}	
			\item VMs befinden sich im gewünschten Zustand
		\end{itemize}  \\
		\hline
		Main Success Story & 
		\begin{enumerate}
			\item Eine VM wird erstellt
			\item Eine VM wird gestartet
			\item VM wird einem Netzwerk zugewiesen
		\end{enumerate}
		\\
		\hline
		Alternative Flows  & 
		\begin{itemize}
			\item[1a.] VM wird gelöscht
		\end{itemize}
	\end{tabularx}
	\caption{UC02 Fully Dressed}
	\label{tab:UC02}
\end{table}

\subsubsection{UC03: Configuration History}
\begin{table}[H]
	\rowcolors{2}{gray!25}{white}
	\centering
	\begin{tabularx}{\textwidth}{l | X}
		Primary Actor      & Administrator        \\
		\hline
		Beschreibung       & Ein Web Interface bietet die Möglichkeit, die Konfigurationen aller Geräte innerhalb einer Fabric anzuzeigen. Des weiteren kann die aktuelle Konfiguration mit älteren Versionen der Konfiguration verglichen werden. \\ 
		\hline
		Stakeholders       &  
		\begin{itemize}	
			\item Administrator
		\end{itemize}              \\
		\hline
		Preconditions      & 
		\begin{itemize}	
			\item Netzwerkgeräte sind erreichbar
		\end{itemize}  \\
		\hline
		Postconditions     & 
		\begin{itemize}	
			\item Konfiguration wird angezeigt
		\end{itemize}  \\
		\hline
		Main Success Story & 
		\begin{enumerate}
			\item Ein Netzwerkgerät wird gewählt
			\item Ein Zeitpunkt wird gewählt
			\item Konfiguration zum gewählten Zeitpunkt kann mit der aktuellen Version verglichen werden
		\end{enumerate}
		\\
		\hline
		Alternative Flows  & 
	\end{tabularx}
	\caption{UC03 Fully Dressed}
	\label{tab:UC03}
\end{table}