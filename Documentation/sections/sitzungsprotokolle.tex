% !TeX encoding = UTF-8
\section{Sitzungsprotokolle}

\subsection{Sitzungsprotokoll 18.09.2018}

\paragraph{Sitzungsteilnehmer}
\begin{itemize}	
	\item Laurent Metzger 
	\item Urs Baumann 
	\item Patrick Mosimann
	\item Sandro Kaspar
	\item Jessica Kalberer
\end{itemize}

\paragraph{Traktanden}
\begin{itemize}
	\item Besprechung genaue Aufgabenstellung und nächste Schritte
	\item Chat für BA
	\item Traktanden vor Meetings Liste an Betreuer? 
	\item ISE Lizenzen abgelaufen
\end{itemize}

\paragraph{Beschlüsse (Diskussion)}
\begin{itemize}	
	\item Heute Kickoff Besprechung für die Bachelorarbeit
	\item Wie krisenresistent ist das Software Defined Netzwerk?
	\item FUB hat folgende Anforderungen und Aufteilung der Bachelorarbeit vorgesehen:
	\begin{itemize}
		\item 10\% der Bachelorarbeit Analyse, wo kritische Stellen der Lösung sind
		\begin{itemize}
			\item Begrenzungen allgemein
			\item DNAC muss keine Internetverbindung haben, da kein Lizenzenforcement bestehen
			\item Wireless Controler erwähnen, dass diese redundant sein müssen, aber in nachfolgender Arbeit eher ausklammern.
			\item Router mit virtuellen Maschinen (berücksichtigen für kritische Stellen, beispielsweise für Spiegelung von Radix Server)
		\end{itemize}
		\item 45\% Analysieren wo die Lösung aufgrund der kritischen Stellen verbessert werden kann
		\begin{itemize}
			\item Transit Fabric oder VRF-Lite vergleichen, was ist besser einzusetzen?
			\item Design anpassen aufgrund der Änderungen
			\item ziemlich sicher hierachisches Modell, da mehrere Fabrics für FUB nötig
			\item für die ganze Planung werden wahrscheinlich Docker Container verwendet
		\end{itemize}
		\item 45\% Implementation mit einem Business Controller (einem abstrahierten Modell)
		\begin{itemize}
			\item 3-4 Use Cases, welche wir von der FUB erhalten
			\item API Controller verwenden: Ziel der Automatisierung, damit nur noch MAC angegeben werden muss und nicht in WebGUI etliche Einstellungen vorgenommen werden können
		\end{itemize}
	\end{itemize}
	\item Patrick wird uns bei Neuerungen informieren, welche von den Releases her für die Umsetzung der Bachelorarbeit relevant sind.
	\item Zielsetzung vorläufig für 24. September 2018: ein laufendes DNA Center, mit allen Geräten sauber provisioned. Darum wird Patrick vor der wöchentlichen Sitzung mit uns zusammensitzen.
	\item Detaillierte Aufgabenstellung wird noch geschrieben, von Cisco und FUB reviewt und uns nächste Woche übergeben	
	\item Traktanden jeweils wieder im Voraus an Betreuen senden
\end{itemize}

\paragraph{Offene Punkte (erledigt vor nächster Sitzung)} \mbox{}
\begin{table}[H]
	\rowcolors{2}{gray!25}{white}
	\centering
	\begin{tabularx}{\textwidth}{X | p{4.5cm}}
		\rowcolor{gray!50}
		\textbf{Was} & \textbf{Verantwortlichkeit} \\
		\hline	
		DNAC Cleanup mit Patrick	& Sandro, Jessica \\	
	\end{tabularx}
	\label{tab:my-label}
\end{table}

\paragraph{Nächster Termin}
\begin{itemize}	
	\item Meeting mit Patrick: 19. September 2018, 09.00 Uhr
	\item Wöchentliches Meeting: 19. September 2018, 10.30 Uhr, 60 Minuten (ausgefallen)
\end{itemize}

\paragraph{Kommende Abwesenheiten} \mbox{}\\
keine

\newpage





\subsection{Sitzungsprotokoll 26.09.2018}

\paragraph{Sitzungsteilnehmer}
\begin{itemize}	
	\item Laurent Metzger 
	\item Urs Baumann 
	\item Patrick Mosimann
	\item Sandro Kaspar
	\item Jessica Kalberer
\end{itemize}

\paragraph{Traktanden}
\begin{itemize}	
	\item Aktueller Stand 
	\begin{itemize}
		\item Anpassungen Architektur
		\item LAN-Automation erfolgreich
		\item Updates der Switches laufend
		\item Fabric noch ausstehend
	\end{itemize}
	\item Aufgabenstellung	
\end{itemize}

\paragraph{Beschlüsse (Diskussion)}
\begin{itemize}	
	\item Saubere Fabric mit Standort Jona fertigstellen
	\item Erster Teil der BA Starten: Analyse
	\begin{itemize}
		\item NTP beachten, was passiert wenn dieser Dienst verlegt wird
	\end{itemize}
	\item Zweiter Teil: Phase Absicherung (bis Ende Oktober erledigt)
	\item Detailbeschreibungen der drei Phasen in Aufgabenstellung ersichtlich. Diese bekommen wir noch im laufe des heutigen Tages.
\end{itemize}

\paragraph{Offene Punkte (erledigt vor nächster Sitzung)} \mbox{}
\begin{table}[H]
	\rowcolors{2}{gray!25}{white}
	\centering
	\begin{tabularx}{\textwidth}{X | p{4.5cm}}
		\rowcolor{gray!50}
		\textbf{Was} & \textbf{Verantwortlichkeit} \\
		\hline	
		Fabric Jona abschliessen & Sandro \& Jessica \\
		Projektplan anhand Aufgabenstellung erarbeiten & Sandro \& Jessica \\
		Teil 1: Analyse Phase starten & Sandro \& Jessica \\
	\end{tabularx}
	\label{tab:my-label}
\end{table}

\paragraph{Nächster Termin}
\begin{itemize}	
	\item Wöchentliches Meeting: 03. Oktober 2018, 10.30 Uhr, 60 Minuten
\end{itemize}

\paragraph{Kommende Abwesenheiten} \mbox{}\\
keine

\newpage





\subsection{Sitzungsprotokoll 03.10.2018}

\paragraph{Sitzungsteilnehmer}
\begin{itemize}	
	\item Laurent Metzger 
	\item Urs Baumann (abwesend)
	\item Patrick Mosimann (abwesend)
	\item Sandro Kaspar
	\item Jessica Kalberer
\end{itemize}

\paragraph{Traktanden}
\begin{itemize}	
	\item Analyse besprechen
	\item VN für Backups ausstehend
	\item Projektplanung
\end{itemize}

\paragraph{Beschlüsse (Diskussion)}
\begin{itemize}	
	\item Analyse und Traktanden werden für Review per Slack an Betreuer gesendet (Terminkonflikt)
\end{itemize}

\paragraph{Offene Punkte (erledigt vor nächster Sitzung)} \mbox{}
\begin{table}[H]
	\rowcolors{2}{gray!25}{white}
	\centering
	\begin{tabularx}{\textwidth}{X | p{4.5cm}}
		\rowcolor{gray!50}
		\textbf{Was} & \textbf{Verantwortlichkeit} \\
		\hline	
		Analyse überarbeiten & Jessica \& Sandro  \\
	\end{tabularx}
	\label{tab:my-label}
\end{table}

\paragraph{Nächster Termin}
\begin{itemize}	
	\item Wöchentliches Meeting: 10. Oktober 2018, 10.30 Uhr, 60 Minuten
\end{itemize}

\paragraph{Kommende Abwesenheiten} \mbox{}\\
keine

\newpage





\subsection{Sitzungsprotokoll 10.10.2018}

\paragraph{Sitzungsteilnehmer}
\begin{itemize}	
	\item Laurent Metzger 
	\item Urs Baumann
	\item Patrick Mosimann
	\item Sandro Kaspar
	\item Jessica Kalberer
\end{itemize}

\paragraph{Traktanden}
\begin{itemize}	
	\item Liefertermin von ECNS 5000
	\item Besprechung der Analyse
	\item Freischaltung Images ISR 4430 und ECNS 5000 (Service Contract Required)
\end{itemize}

\paragraph{Beschlüsse (Diskussion)}
\begin{itemize}
	\item Analyse Besprechnung
	\begin{itemize}
		\item SDA Architektur und Design
		\begin{itemize}
			\item Informationen mit neustem Release 1.2.5 updaten
			\item Maximum Scale - nur 20 Fabric Domains? Es gibt mehr Kantone!
		\end{itemize}
		\item Cisco Severity and Escalation Guidelines implementieren
		\begin{itemize}
			\item Eigene Severity erstellen (Beispielsweise ist in den Cisco Guidelines der DNA Ausfall erst bei Severity 3) 
			\item Verfügbarkeiten mit Severity ergänzen
		\end{itemize}
		\item Verfügbarkeiten
		\begin{itemize}
			\item LISP Map Server / Control Plane Node
			\begin{itemize}
				\item LISP abhängig von der Platzierung (ECNS Control Plane virtualisieren, gibt es Möglichkeiten?), Phyton Script das eventuell Map Cache statisch konfiguriert
				\item Severity 1
				\item dezentrale Lösung wenn möglich
			\end{itemize}
			\item ISE / Radius
			\begin{itemize}
				\item Was passiert mit schon bestehenden angemeldeten Personen bei Ausfall?
				\item Annahmen treffen und diese in nächsten Teilen der Arbeit testen
				\item Severity 1
				\item dezentrale Lösung
			\end{itemize}
			\item SGT Access List
			\begin{itemize}
				\item Detaillierter beschreiben was in welchen Fällen passiert
				\item Beispiel:  SGT Access Listen werden nur auf Control Node gespeichert (nur wegen SXP)
				\item Beispiel: sobald Client weg ist, wird dieser wieder aus der SGT Access List gelöscht? Was passiert bei nächster Anmeldung? 
				\item Severity 1
				\item dezentrale Lösung möglich?
			\end{itemize}
			\item Border Node
			\begin{itemize}
				\item Das meiste sollte funktionieren, solange lokaler ISE/Radius etc. vorhanden
				\item keine dezentrale Lösung nötig
				\item Wie funktioniert die Redundanz?
				\item Wie ist die Konvergenz (ISIS/BGP)? Gibt es eine Optimierung?
				\item 
			\end{itemize}
			\item Fusion Router
			\begin{itemize}
				\item Kein Leaking zwischen VRFs?
				\item Herr Metzger kennt sich mit dieser Materie sehr gut aus. Nicht zu viel Zeit verlieren und direkt mit ihm die Konfiguration anschauen.
				\item Urs kann bei Fragen beim Orchestration Tool konsultiert werden.
				\item dezentrale Lösung
			\end{itemize}
			\item DHCP, NTP, DNS
			\begin{itemize}
				\item DNS UND NTP können mehrere angegeben werden, so das im Notfall einfach der andere Server genommen wird
				\item Severity 2
			\end{itemize}
			\item Lizenzen
			\begin{itemize}
				\item Severity 4 - falls diese aber enforced werden wird es Severity 1 oder Severity 2
			\end{itemize}
		\end{itemize}
	\end{itemize}
	\item Projektmanagement
	\begin{itemize}
		\item Risiken gut, aber im Bild sind R5 und R6 nicht ersichtlich
	\end{itemize}
	\item Liefertermin von ECNS 5000 wird noch abgeklärt
	\item nächste Schritte
	\begin{itemize}
		\item Verbindungen mit VRF-Lite und Transit Fabric (VXLAN)
		\item VRF Route Leaking einrichten (da sonst keine Kontrolle wenn alles in einer Routing Tabelle)
		\item anpassung Architektur - Router als zweiter Legacy einrichten
		\item Bei FUB ist die Telefonie zentral - sie müssen keine IP Connectivität haben
	\end{itemize}
	\item Dokument wieder an Herrn Metzger und Urs senden, damit sie Reviewen können
\end{itemize}


\paragraph{Offene Punkte (erledigt vor nächster Sitzung)} \mbox{}
\begin{table}[H]
	\rowcolors{2}{gray!25}{white}
	\centering
	\begin{tabularx}{\textwidth}{X | p{4.5cm}}
		\rowcolor{gray!50}
		\textbf{Was} & \textbf{Verantwortlichkeit} \\
		\hline	
		Analyse überarbeiten & Jessica \& Sandro  \\
		Dokument für Review an Betreuer senden & Jessica \\
		Teil 2: Absicherung vorbereiten & Jessica \& Sandro \\
	\end{tabularx}
	\label{tab:my-label}
\end{table}

\paragraph{Nächster Termin}
\begin{itemize}	
	\item Wöchentliches Meeting: 17. Oktober 2018, 10.30 Uhr, 60 Minuten
\end{itemize}

\paragraph{Kommende Abwesenheiten} \mbox{}\\
keine

\newpage





\subsection{Sitzungsprotokoll 17.10.2018}

\paragraph{Sitzungsteilnehmer}
\begin{itemize}	
	\item Laurent Metzger 
	\item Urs Baumann
	\item Patrick Mosimann
	\item Sandro Kaspar
	\item Jessica Kalberer
\end{itemize}

\paragraph{Traktanden}
\begin{itemize}	
	\item 
	\item 
	\item 
\end{itemize}

\paragraph{Beschlüsse (Diskussion)}
\begin{itemize}	
	\item 
	\item 
	\begin{itemize}
		\item 
		\item
	\end{itemize}
	\item 
	\item 
	\begin{itemize}
		\item 
		\item 
	\end{itemize}
\end{itemize}

\paragraph{Offene Punkte (erledigt vor nächster Sitzung)} \mbox{}
\begin{table}[H]
	\rowcolors{2}{gray!25}{white}
	\centering
	\begin{tabularx}{\textwidth}{X | p{4.5cm}}
		\rowcolor{gray!50}
		\textbf{Was} & \textbf{Verantwortlichkeit} \\
		\hline	
		 & Jessica \& Sandro  \\
		 & Jessica \& Sandro \\
	\end{tabularx}
	\label{tab:my-label}
\end{table}

\paragraph{Nächster Termin}
\begin{itemize}	
	\item Wöchentliches Meeting: 31. Oktober 2018, 10.30 Uhr, 60 Minuten
\end{itemize}

\paragraph{Kommende Abwesenheiten} \mbox{}\\
keine