\section{Ergebnisdiskussion}

Die Analyse der kritischen Komponenten wurde abgeschlossen und diese sind identifiziert. Dazu wurden die kritischen Komponenten nach ihrer Prorität gewichtet. 

Während der Absicherung wurden die kritischen Komponenten genauer analysiert, die Möglichkeiten zur Absicherung abgewägt und eine Empfehlung für ein krisensicheres Deployment erarbeitet. Anschliessend wurden die Empfehlungen wo möglich im eigenen Lab angewendet und somit die meisten Komponenten und Verbindungen redundant ausgelegt.

Die Abstrahierung konnte teilweise umgesetzt werden. Bei der Entwicklung des Orchestrierungstool kam es zu Problemen mit den DNA Center APIs. Diese waren teilweise falsch oder unvollständig dokumentiert. Zudem mussten viele API Endpoints verwendet werden, die in keiner Dokumentation enthalten sind.
Aus diesem Grund nahm die Entwicklung des Tools mehr Zeit als geplant in Anspruch und es konnten nicht alle Use Cases komplett abgedeckt werden.

