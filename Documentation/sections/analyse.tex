\section{Analyse}
Nachfolgend werden alle kritischen Elemente der SDA Lösung analysiert. Dazu gehören beispielsweise Dienste, wie die LISP Datenbank, Radius, SGT Access List. und 

Das Ziel ist die komplette Analyse der SDA Lösung und die Identifizierung der kritischen Elemente der Verfügbarkeit (LISP Database, Radius, SGT Access-list etc.) und der Network Services (NTP, DNS, Lizenzen etc.).

\subsection{SDA Architektur und Design}
Unsere Architektur der Lab Umgebung, welche in der Studienarbeit erarbeitet und nun in der Bachelorarbeit noch erweitert wurde, sieht zur Zeit wie folgend aus:

\begin{figure}[H]
	\centering
	\includegraphics[width=1\linewidth]{img/Architecture/LabNetworkArchitecture-25-09}
	\caption{Architektur}
	\label{fig:Architektur}
\end{figure}

Die Analyse wird auf dieser Architektur aufbauen und die von der FUB gegebenen Grössenordnung berücksichtigen.

Nachfolgend werden die aktuell von Cisco empfohlenen Design Entscheidungen, Grössenüberlegungen, sowie die Skalierungen gezeigt. Aktuelle und ausführliche Informationen hierzu, können direkt im SDA Design Guide von Cisco \cite{sda-designguide} eingesehen werden. 

\subsubsection{Entscheidungen}
Die Design Entscheidungen beruhen auf den aktuellen Empfehlungen von Cisco, welche sich laufend Ändern können.

\begin{figure}[H]
	\centering
	\includegraphics[width=1\linewidth]{img/Analyse/SDAswitchingplatformsanddeploymentcapabilities}
	\caption{SDA switching platforms and deployment capabilities \cite{sda-designguide}}
	\label{fig:SDA switching platforms and deployment capabilities}
\end{figure}

\begin{figure}[H]
	\centering
	\includegraphics[width=1\linewidth]{img/Analyse/SDAroutingandwirelessplatformsanddeploymentcapabilities}
	\caption{SDA routing and wireless platforms and deployment capabilities \cite{sda-designguide}}
	\label{fig:SDA routing and wireless platforms and deployment capabilities}
\end{figure}

\subsubsection{Grössenüberlegungen}
Es ist zusätzlich zu den oben in den Design Entscheidungen empfohlenen Plattformen wichtig, die nachfolgenden getesteten Werte für Cisco Validated Design (CVD) sowie die Versionshinweise zu Hardware und Software für eine Bereitstellung zu berücksichtigen.

Diese Grössenüberlegungen sind besonders für grosse Unternehmen wichtig, da diese zur Zeit nur bis zur eines gewissen Anzahl von Cisco getestet wurden und im schlechtesten Fall nicht höher unterstützt werden.


\begin{figure}[H]
	\centering
	\includegraphics[width=1\linewidth]{img/Analyse/CVD-DNACmanagementofSDA}
	\caption{DNAC management of SDA \cite{sda-designguide} }
	\label{fig:DNAC management of SDA}
\end{figure}

\begin{figure}[H]
	\centering
	\includegraphics[width=1\linewidth]{img/Analyse/CVD-DNACwithSDAperFabricDomain}
	\caption{DNAC with SDA per each fabric domain \cite{sda-designguide} }
	\label{fig:DNAC with SDA per each fabric domain}
\end{figure}

\begin{figure}[H]
	\centering
	\includegraphics[width=1\linewidth]{img/Analyse/CVD-SDAvirtualnetworksbyplatformandrole}
	\caption{SDA virtual networks by platform and role \cite{sda-designguide} }
	\label{fig:SDA virtual networks by platform and role}
\end{figure}


\subsubsection{Skalierungen}
Folgende maximalen Skalierungen sollten für grosse Organisationen besonders berücksichtigt werden. Diese Daten sollten bei der Auswahl von Plattformen, die während der Planung für das aktuelle und zukünftige Wachstum des Netzwerks verwendet werden, berücksichtigt werden.

Die DNAC Nummern sind pro Instanz, bei denen es sich um ein DNAC mit einem einzelnen Server oder einem DNA-Cluster mit drei Servern handeln kann. Die maximalen Zahlen sind entweder die absoluten Grenzen der Plattform oder die empfohlenen Höchstgrenzen aktueller Tests einer einzelnen Plattform.

\begin{figure}[H]
	\centering
	\includegraphics[width=1\linewidth]{img/Analyse/MaxScale-DNAC}
	\caption{DNAC Maximum Scale Recommendations \cite{sda-designguide} }
	\label{fig:DNAC Maximum Scale RecommendationsA}
\end{figure}

\begin{figure}[H]
	\centering
	\includegraphics[width=1\linewidth]{img/Analyse/MaxScale-EdgeNode}
	\caption{Edge Node Maximum Scale Recommendations \cite{sda-designguide} }
	\label{fig:Edge Node Maximum Scale RecommendationsA}
\end{figure}

\begin{figure}[H]
	\centering
	\includegraphics[width=1\linewidth]{img/Analyse/MaxScale-BorderNode}
	\caption{Border Node Maximum Scale Recommendations \cite{sda-designguide} }
	\label{fig:Border Node Maximum Scale RecommendationsA}
\end{figure}




\subsection{Verfügbarkeit}
Die Verfügbarkeit der nachfolgenden Dienste sind für den Betrieb eines SDA mit dem DNAC wichtig, um die volle Funktion des Netzwerkes bereitzustellen. Darum ist es von Vorteil, wenn die kritischen Server und Dienste redundant ausgelegt sind.


\subsubsection{LISP Database}

\subsubsection{Radius}

\subsubsection{SGT Access List}
\paragraph{Beschreibung}
Der Sicherheitsgruppen-Tag (SGT) weist jeder Sicherheitsgruppe eine eindeutige 16-Bit-Sicherheitsgruppennummer zu, deren Geltungsbereich in einer TrustSec-Domäne global ist. Die Nummern werden automatisch generiert, wenn eine SGT auf dem ISE erstellt wird. Die Anzahl der Sicherheitsgruppen im Switch ist auf die Anzahl der authentifizierten Netzwerkeinheiten beschränkt.

Die SGT ermöglicht es dem ISE, Richtlinien für die Zugriffssteuerung durchzusetzen, indem es dem Endgerät ermöglicht, auf die SGT zu reagieren, um den Datenverkehr zu filtern.

Die Sicherheitsgruppen-Zugriffssteuerungsliste (SGACL) ermöglicht die Steuerung des Zugriffs und der Berechtigungen basierend auf den zugewiesenen SGTs.

Ablauf: In TrustSec-Netzwerken werden Pakete am Egress gefiltert, nicht der Eintritt in das Netzwerk. Bei der TrustSec-Endpunktauthentifizierung wird einem Host, der auf die TrustSec-Domäne (Endpunkt-IP-Adresse) zugreift, über DHCP-Snooping und IP-Geräteverfolgung ein Sicherheitsgruppen-Tag (SGT) am Zugriffsgerät zugeordnet. Das Zugriffsgerät überträgt diese Zuordnung über Hardware-fähige SXP zu TrustSec-Ausgangsgeräten, die eine kontinuierlich aktualisierte Tabelle von Quell-IP zu SGT-Bindungen verwalten. Pakete werden auf dem Ausgang durch die Hardware-fähigen TrustSec-Geräte gefiltert, indem die Sicherheitsgruppe ACLS (SGACLs) angewendet wird.

\paragraph{Impact bei Ausfall}
DNAC: Auf dem DNAC sind die Sicherheitsgruppen nicht mehr verfügbar und können auch nicht neu erstellt werden.

\paragraph{Mögliche Ansätze für bessere krisenresistenz}
Switch-Stack
Redundanz des ISE
Backup, damit schneller Restore möglich


\subsubsection{NTP}
\paragraph{Beschreibung}
Das Network Time Protocol (NTP) ist ein Standard zur Synchronisierung der Zeit auf den verschiedenen Systemen.

\paragraph{Impact bei Ausfall}
Zertifikate nicht mehr gültig oder Überprüfung schlägt fehl.
Logs fehlerhaft, da Timestamps nicht richtig. Fehlersuche erschwert sich enorm.

\paragraph{Mögliche Ansätze für bessere krisenresistenz}
Mehrere NTP-Server angeben - Cisco empfiehlt drei NTP Server anzugeben

\subsubsection{DNS}
\paragraph{Beschreibung}
Der Domain Name Service (DNS) ost eine wesentliche und oft unterschätzte Komponente in einem Netzwerk. Es ist wichtig, dass DNS im Netzwerk korrekt funktioniert und jederzeit zur Verfügung steht.

Es ist also von Vorteil wenn zur Verbeugung eines Ausfalls des DNS, der Server bzw. der Dienst redundant ausgelegt ist.

\paragraph{Impact bei Ausfall}
Namensauflösung der Geräte funktioniert nicht mehr oder ist viel zu langsam.


\paragraph{Mögliche Ansätze für bessere krisenresistenz}
DNS Server redundant
Mehrere DNS Server angeben

"DNS Server IP Addresses (Required): The IP address and subnet mask for one or more of your network's preferred DNS servers. During configuration, you can specify multiple DNS server IP addresses and netmasks by entering them as a space-separated list. Note that, due to an unresolved bug, customers currently cannot change the list of DNS servers after configuration. If you find that you must change the list of DNS server IPs after configuration, contact the Cisco Technical Assistance Center (TAC)."

\subsubsection{Lizenzen}
\paragraph{Beschreibung}
Die Lizenzen der Geräte können über das DNAC verwaltet werden. Dazu ist jedoch ein Cisco Smart Account nötig, der über alle Lizenzen verfügt. Sind die Lizenzen einmal auf dem DNAC synchronisiert und ersichtlich, so muss keine Internetverbindung bestehen, da kein Lizenzenforcement besteht.

Die Lizenzen für die einzelnen Switche müssen beim Kauf unbedingt beachtet werden. Jedoch können auf den Switches die Lizenzen auf einer Evaluation Basis manuell nachgerüstet werden (beispielsweise von IP Base auf IP Services).

\paragraph{Impact bei Ausfall}
Keine, da Lizenzen nicht enforced werden und Lizenzen somit weiterlaufen.

\paragraph{Mögliche Ansätze für bessere krisenresistenz}
Richtige Lizenzen für alle Geräte von Anfang an einplanen.

\subsection{Hardware}
\paragraph{Beschreibung}
Die Hardware ist egal ob es das DNAC, ein anderer Server oder ein Netzwerkgerät ist, wohl die wichtigste Komponente, das ein Netzwerk reibungslos läuft.

\paragraph{Impact bei Ausfall}
DNAC, ISE, Netzwerk nicht mehr verfügbar

\paragraph{Mögliche Ansätze für bessere krisenresistenz}
Bei der Hardware ist zu beachten, dass sowohl der Strom, als auch die Netzwerkverbindungen redundant vorhanden sind. Ebenfalls sollte die Hardware bei einem Stromausfall durch eine USV eine gewisse Zeit weiter betrieben werden können. Ein Backup für den Notfall ist immer zu empfehlen, damit bei einem Hardwaredefekt ein schneller Restore möglich ist.
Ersatzgeräte sind von Vorteil.

\paragraph{Wireless Controller}
\paragraph{Beschreibung}
Die Wireless Controller müssen redundant ausgelegt sein. Es wird in der nachfolgenden Arbeit aber nicht mehr genauer auf Wireless Controller eingegangen, da diese in der Bachelorarbeit nicht integriert werden.

\paragraph{Impact bei Ausfall}
Wireless steht nicht mehr zur Verfügung. Sensoren die über Wireless laufen oder Geräte ohne Netzwerkanschluss können nicht mehr genutzt werden.

\paragraph{Mögliche Ansätze für bessere krisenresistenz}
Wireless Controller redundant auslegen
Ersatzgeräte, damit bei Hardwaredefekt schneller Austausch möglich
