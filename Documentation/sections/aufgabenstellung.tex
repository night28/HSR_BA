\section{Aufgabenstellung}
Dies ist die initiale Aufgabenstellung, welche zu Beginn der Bachelorarbeit vorlag. 

\subsection{Erster Teil (~10\%): Phase Analyse}
Ziel: Sämtliche Analyse der SD-A Lösung und Identifizierung der kritischen Elemente der Verfügbarkeit (LISP Database, Radius, SGT Access-list,…) und der Network Services (NTP, DNS, Lizenzen, ….)
\begin{itemize}
	\item Die Clients werden am Netz mit dot1x authentifiziert und mit Wired Access verbunden. (Wireless Access ist nicht Teil der BA)
\end{itemize}


\subsection{Zweiter Teil (~45\%): Phase Absicherung}
Ziel: Erstellung von Technische Design und Ansätze um die SD-A Lösung Krisenresistenter zu machen.
\begin{itemize}
	\item Die Arbeiten umfassen:
	\begin{itemize}
		\item Analyse und Gegenüberstellung der mögliche Optionen für die Verbindungen von zwei SD-A Fabric (VRF-Lite, Transit Fabric, …)
		\item Empfehlungen für das Deployment eines krisensicheren SD-A
		\item Aufbau eines Pilots mit mindestens zwei Standorte
		\item Durchführung der definierten Vorgaben im Pilot
	\end{itemize}
	\item Ein ECNS 5000 wird zur Verfügung gestellt und kann in der krisensicheren Lösung integriert sein.
\end{itemize}


\subsection{Dritter Teil (~45\%): Phase Abstrahierung}
Ziel: Entwicklung eines Operations Orchestrators für 2-3 Use Cases. 
\begin{itemize}
	\item Die Use cases werden von Betreuern und Industiepartner festgelegt.
	\item Der Operations Orchestrator wird bestimmte Operations Tätigkeiten vereinfachen und wird mit DNA-Center, ISE und dem Fusion-Router interagieren. (Wenn zwei VRFs zusammen kommunizieren werden die Fabric und der Fusion Router gleichzeitig konfiguriert)
	\item Idealerweise gibt es ein Web Interfaces für Demozweck.
\end{itemize}

Notwendige Kenntnisse: Routing\&Switching, VXLAN overlays, Network Services, objektorientierte programmierung (python bevorzugt)


