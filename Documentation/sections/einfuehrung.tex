\section{Einführung}

Wie schon erwähnt ist diese Bachelorarbeit eine Folgearbeit der Studienarbeit Software-Defined Netzwerk im Campus Bereich \cite{studienarbeit}, was bedeutet das in dieser Arbeit nicht mehr auf die grundlegenden Technologien eingegangen wird. Es werden jedoch Technologien, welche für das krisenresistente Netzwerk besonders relevant sind, detaillierter analysiert und beschrieben. \\

In der Studienarbeit hat sich relativ klar herauskristallisiert, dass bei einem so grossen Organisation wie der FUB, ein besonderes Augenmerk auf die Skalierbarkeit gelegt werden muss. Auf Grund der aufgetretenen Probleme und Bugs, konnten diese Anforderungen jedoch nur kurz angeschnitten werden. Im Verlauf der Studienarbeit wurde eine laufende Fabric mit der höchsten Priorität definiert. Darauf folgend konnten die vom Industriepartner definierten Use Cases wenn möglich praktisch und sonst theoretisch abgedeckt werden. Die Use Cases lauteten folgendermaßen \cite{studienarbeit}:

\begin{itemize}
	\item Definition von Benutzerprofilen
	\item Benutzermobilität
	\item Reporting der Netzwerkaktivitäten
	\item Degradation der Infrastruktur
	\item Backup und Restore
	\item Anbindung an externe Systeme, wie ISE und Infoblox
\end{itemize}

Wie bei neuerer Software üblich, waren auch in DNA Center noch verhältnismässig viele
Bugs vorhanden. Die Bugs wurden in der Studienarbeit dokumentiert und jeweils an die Cisco
Experten weitergeleitet. Teilweise konnten die Bugs durch neue Releases schon behoben werden, wiederum andere sind noch ausstehend.

\subsection{Erkentnisse aus der Studienarbeit}
Erkentnisse aus der Studienarbeit - Anforderungen welche in der Studienarbeit


\subsection{Krisenresistentes Software Defined Netzwerk}
Nun soll in dieser Bachelorarbeit ein krisenrstistentes Netzwerk erstellt werden, welches Technologien wie das DNA Center, ISE, Infoblox, LISP






