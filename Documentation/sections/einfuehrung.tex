\section{Einführung}

Wie schon erwähnt ist diese Bachelorarbeit eine Folgearbeit der Studienarbeit Software-Defined Netzwerk im Campus Bereich \cite{studienarbeit}, was bedeutet, dass in dieser Arbeit nicht mehr auf die grundlegenden Technologien eingegangen wird. Es werden jedoch Technologien, welche für das krisenresistente Netzwerk besonders relevant sind, detaillierter analysiert und beschrieben. 

\subsection{Erkenntnisse aus der Studienarbeit}

In der Studienarbeit hat sich klar gezeigt, dass bei einer so grossen Organisation wie der FUB ein besonderes Augenmerk auf die Skalierbarkeit gelegt werden muss. Auf Grund der aufgetretenen Probleme und Bugs, konnten diese Anforderungen jedoch nur kurz angeschnitten werden. Im Verlauf der Studienarbeit wurde eine laufende Fabric mit der höchsten Priorität definiert. Darauf folgend konnten die vom Industriepartner definierten Use Cases praktisch und wenn nicht anders möglich theoretisch abgedeckt werden. Die Use Cases lauteten folgendermassen \cite{studienarbeit}:

\begin{itemize}
	\item Definition von Benutzerprofilen
	\item Benutzermobilität
	\item Reporting der Netzwerkaktivitäten
	\item Degradation der Infrastruktur
	\item Backup und Restore
	\item Anbindung an externe Systeme, wie ISE und Infoblox
\end{itemize}


Wie bei neuerer Software üblich, waren auch im DNA Center noch verhältnismässig viele Bugs vorhanden. Die Bugs wurden in der Studienarbeit dokumentiert und jeweils an die Cisco Experten weitergeleitet. Teilweise konnten die Bugs durch neue Releases schon behoben werden, wiederum andere sind noch ausstehend. 


\subsection{Krisenresistentes Software Defined Netzwerk}
Nun soll in dieser Bachelorarbeit ein krisenresistentes Software Defined Netzwerk erstellt werden, welches mit Hilfe des DNA Centers und Technologien wie VXLAN und LISP umgesetzt wird. Diese Technologien wurden in der Studienarbeit genauer dokumentiert \cite{studienarbeit}. In der Bachelorarbeit wird zum einen ein möglichst krisenresistentes Netzwerk erstellt und für dies eine Empfehlung für das Deployment vorgestellt. Zum anderen sollen die APIs des DNA Centers genauer untersucht und in einem Orchestrierungstool verwendet werden. In der Studienarbeit waren bis anhin nur wenige Informationen zur API verfügbar, welche nun mit dem Release 1.2.5 aber relativ gut abgedeckt sein sollten. In der Analyse der ganzen SDA Lösung soll nun auch die Skalierbarkeit genauer untersucht werden. Es ist zu klären ob die aktuellen Skalierbarkeitsempfehlungen von Cisco, für so eine grosse Organisation wie der FUB genüben. Mit einem ECNS 5400 soll zudem ein autonomer Standort mit den nötigen Diensten wie Radius, DHCP, DNS und NTP simuliert werden können.





