\section{Abstract} 

\subsection{Aufgabenstellung} 
Die Aufgabe dieser Bachelorarbeit war es, eine verteilte, krisenresistente Software-Defined Access (SDA) Lösung für die Führungsunterstützungsbasis der Armee (FUB) zu erstellen. Dies basierend auf dem DNA Center von Cisco, welches bereits in einer Studienarbeit in Betrieb genommen wurde. In einem ersten Schritt wurden die kritischen Komponenten der Lösung identifiziert und Ansätze entwickelt, wie diese Komponenten abgesichert werden können. Im darauffolgenden Schritt wurden diese Ansätze konkretisiert und umgesetzt. Der dritte Teil der Aufgabe war, ein Orchestrierungstool zu entwickeln, welches die gängigsten Operations Tasks vereinfacht und teilweise automatisiert. Folgende Use Cases soll die Anwendung abdecken:
\begin{itemize}
	\item Netzwerke erstellen und verwalten
	\begin{itemize}
		\item Netzwerke im DNA Center anlegen
		\item Konfiguration der Fusion Router
		\item Access Policies zwischen den Netzwerken konfigurieren
	\end{itemize}
	\item Management der virtuellen Maschinen
	\item Anzeigen aller Konfigurationen und der dazugehörigen History
\end{itemize}


\subsection{Vorgehen}
In dieser Bachelorarbeit wurden grundlegende Ansätze zum Betrieb einer SDA Lösung erarbeitet, die in Zukunft weiterentwickelt werden können. Zu Beginn wurde das Lab, welches in der Studienarbeit erstellt wurde analysiert und die kritischen Komponenten, die für den Betrieb der Umgebung nötig sind, identifiziert. Zudem wurden Ideen entwickelt, wie diese kritischen Komponenten abgesichert und redundant betrieben werden können, sodass der Betrieb auch bei Ausfall von Komponenten oder Verbindungen gewährleistet ist. Ebenfalls wurde die Architektur der Umgebung analysiert und verbessert, sodass die Anforderungen an die Verfügbarkeit erfüllt sind. In einem nächsten Schritt wurden die zuvor erarbeiteten Ideen in der Testumgebung praktisch umgesetzt und getestet. Für viele Services, die in der verteilten Umgebung an allen Standorten verfügbar sein müssen, wurde eine Virtualisierungsplattform von Cisco verwendet. Im dritten Teil der Arbeit wurde ein Orchestrierungstool entwickelt. Mit Hilfe von diesem kann ein Operator die gängigsten Aufgaben über ein Web-Interface ausführen. Das Tool automatisiert zudem Arbeitsschritte, die im DNA Center noch nicht abgedeckt sind und manuell ausgeführt werden müssten. 

\subsection{Fazit}
Es kann gesagt werden, dass die zentrale Struktur einer SDA Lösung die Sicherstellung der Verfügbarkeit in einer verteilten Umgebung erschwert. Die kritischen Komponenten, die in der Analyse definiert wurden, konnten dennoch zum grössten Teil erfolgreich abgesichert werden. Allerdings ermöglichen die zentralen Controller dieser Struktur ein programmatisches und automatisiertes Management der Infrastruktur über APIs. Nicht nur aus diesem Grund ist das Software-Defined Netzwerk (SDN) definitiv das Netzwerk der Zukunft. 

