\section{Feature Requests und Bugs}
Im Folgenden werden Feature Requests und Bugs aufgeführt, die während dieser Arbeit aufgetreten sind. Diese wurden mittesl "Make a Wish" Funktion des DNA Centers an Cisco gemeldet. Alle erwähnten Punkte beziehen sich auf das DNA Center in Version 1.2.5.


\newcommand{\bugreport}[9]{
\subsection{#1}
	\begin{table}[H]
		\rowcolors{2}{gray!25}{white}
		\centering
		\begin{tabularx}{\textwidth}{| l | X |}
			\hline
			Komponente   & #2       \\
			\hline
			Priorität   & #3       \\
			\hline
			Beschreibung   & #4  \\ 
			\hline
			Konsequenzen   & #5  \\ 
			\hline
			Workaround & #6 \\
			\hline
			Reproduzieren & #7	\\
			\hline
			Reporter  & #8 \\
			\hline
			Feedback Cisco & #9 \\
			\hline
		\end{tabularx}
		\caption{Bug: #1}
	\end{table}
}

\newcommand{\featureRequest}[5]{
	\subsection{#1}
	\begin{table}[H]
		\rowcolors{2}{gray!25}{white}
		\centering
		\begin{tabularx}{\textwidth}{| l | X |}
			\hline
			Komponente   & #2       \\
			\hline
			Beschreibung   & #3  \\ 
			\hline
			Reporter  & #4 \\
			\hline
			Feedback Cisco & #5 \\
			\hline
		\end{tabularx}
		\caption{Feature Request: #1}
	\end{table}
}

\featureRequest
% Titel
{Template Zuweisung}
% Komponente
{\textit{Design $\rightarrow$ Network Profiles}}
% Beschreibung
{Templates können nur Gerätetypen (z.Bsp. Switches oder Router) zugewiesen werden. Es wäre wünschenswert, wenn diese auch verschiedenen Rollen zugewiesen werden könnten. Beispielsweise allen Border Nodes.
}
% Reporter
{Sandro Kaspar}
% Feedback Cisco
{}

\featureRequest
% Titel
{Template Versionierung}
% Komponente
{\textit{Template Editor $\rightarrow$ Create Template}}
% Beschreibung
{Templates können versioniert werden. Allerdings lassen sich ältere Versionen nur anschauen. Folgende Funktionen wären hilfreich:
\begin{itemize}
	\item Restore alter Templateversionen
	\item Diffs zwischen verschiedenen Versionen anzeigen
\end{itemize}
}
% Reporter
{Sandro Kaspar}
% Feedback Cisco
{}

\bugreport
% Titel
{Beispielbug}
% Komponente
{\textit{Einstellungen $\rightarrow$ System Settings $\rightarrow$ Backup \& Restore}}
% Dringlichkeit
{Hoch}
% Beschreibung
{Nach der Eingabe der Backup Server Einstellungen und dem Klick auf \textit{Apply} geschieht nichts. Nach einer Weile stürzen immer mehr Docker Container ab, bis das DNA Center nicht mehr gebraucht werden kann. Ein Neustart des DNA Centers ist erforderlich.
\textbf{Nachtrag:} Nach mehreren Versuchen mit verschiedenen SSH Servern hat es geklappt. Über die genaue Ursache kann keine Aussage gemacht werden.
}
% Konsequenzen
{Beim Ausfall der Appliance können die Einstellungen nicht wiederhergestellt werden.}
% Workaround
{Keiner}
%Reproduzieren
{
	\begin{enumerate}
		\item \textit{Settings $\rightarrow$ System Settings $\rightarrow$ Backup \& Restore}
		\item Im Popup \textit{Configure} wählen. 
		\item SSH Servereinstellungen eingeben
		\item \textit{Apply} drücken. 
	\end{enumerate}
}
% Reporter
{Sandro Kaspar}
% Feedback Cisco
{}

