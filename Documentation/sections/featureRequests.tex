\section{Feature Requests und Bugs}
Im Folgenden werden Feature Requests und Bugs aufgeführt, die während dieser Arbeit aufgetreten sind. Diese wurden mittesl "Make a Wish" Funktion des DNA Centers an Cisco gemeldet. Alle erwähnten Punkte beziehen sich auf das DNA Center in Version 1.2.5.


\newcommand{\bugreport}[9]{
\subsection{#1}
	\begin{table}[H]
		\rowcolors{2}{gray!25}{white}
		\centering
		\begin{tabularx}{\textwidth}{| l | X |}
			\hline
			Komponente   & #2       \\
			\hline
			Priorität   & #3       \\
			\hline
			Beschreibung   & #4  \\ 
			\hline
			Konsequenzen   & #5  \\ 
			\hline
			Workaround & #6 \\
			\hline
			Reproduzieren & #7	\\
			\hline
			Reporter  & #8 \\
			\hline
			Feedback Cisco & #9 \\
			\hline
		\end{tabularx}
		\caption{Bug: #1}
	\end{table}
}

\newcommand{\featureRequest}[5]{
	\subsection{#1}
	\begin{table}[H]
		\rowcolors{2}{gray!25}{white}
		\centering
		\begin{tabularx}{\textwidth}{| l | X |}
			\hline
			Komponente   & #2       \\
			\hline
			Beschreibung   & #3  \\ 
			\hline
			Reporter  & #4 \\
			\hline
			Feedback Cisco & #5 \\
			\hline
		\end{tabularx}
		\caption{Feature Request: #1}
	\end{table}
}

\featureRequest
% Titel
{Template Zuweisung}
% Komponente
{\textit{Design $\rightarrow$ Network Profiles}}
% Beschreibung
{Templates können nur Gerätetypen (z.Bsp. Switches oder Router) zugewiesen werden. Es wäre wünschenswert, wenn diese auch verschiedenen Rollen zugewiesen werden könnten. Beispielsweise allen Border Nodes.
}
% Reporter
{Sandro Kaspar}
% Feedback Cisco
{}

\featureRequest
% Titel
{Template Versionierung}
% Komponente
{\textit{Template Editor $\rightarrow$ Create Template}}
% Beschreibung
{Templates können versioniert werden. Allerdings lassen sich ältere Versionen nur anschauen. Folgende Funktionen wären hilfreich:
\begin{itemize}
	\item Restore alter Templateversionen
	\item Diffs zwischen verschiedenen Versionen anzeigen
\end{itemize}
}
% Reporter
{Sandro Kaspar}
% Feedback Cisco
{}

\bugreport
% Titel
{Upload Image}
% Komponente
{\textit{Design $\rightarrow$ Image Repository $\rightarrow$ Import}}
% Dringlichkeit
{Mittel}
% Beschreibung
{Nach dem Uploaden eines Images, wird dieses nicht im WebGui angezeigt. Erst nach mehreren Minuten erscheint das zuvor hochgeladene Image.
}
% Konsequenzen
{Der User geht davon aus, dass der Upload fehlgeschlagen ist und versucht dies erneut.}
% Workaround
{Keiner}
%Reproduzieren
{
	\begin{enumerate}
		\item \textit{Design $\rightarrow$ Image Repository $\rightarrow$ Import}
		\item File auswählen
		\item Warten bis der Upload abgeschlossen ist
		\item Image wird nicht angezeigt
		\subitem Nach mehreren Minuten erscheint das Image
	\end{enumerate}
}
% Reporter
{Sandro Kaspar}
% Feedback Cisco
{}

When a device first appears in inventory, information like hostname or OS Version are displayed as "null". Show correct data or something like "not available".

\bugreport
% Titel
{Claim Device}
% Komponente
{\textit{Provision $\rightarrow$ Inventory}}
% Dringlichkeit
{Niedrig}
% Beschreibung
{Nach dem Hinzufügen oder Claimen eines Devices werden verschiedene Informationen als "null" angezeigt. Hier sollten die korrekten Informationen oder falls noch nicht verfügbar einfach nichts angezeigt werden.
}
% Konsequenzen
{Unschoönes UI}
% Workaround
{Keiner}
%Reproduzieren
{
	\begin{enumerate}
		\item \textit{Provision $\rightarrow$ Inventory $\rightarrow$ Unclaimed Devices $\rightarrow$ Clain Device}
	\end{enumerate}
}
% Reporter
{Sandro Kaspar}
% Feedback Cisco
{}