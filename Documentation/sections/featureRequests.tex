\section{Feature Requests und Bugs}
Im Folgenden werden Feature Requests und Bugs aufgeführt, die während dieser Arbeit aufgetreten sind. Diese wurden mittesl "Make a Wish" Funktion des DNA Centers an Cisco gemeldet. Alle erwähnten Punkte beziehen sich auf das DNA Center in Version 1.2.5.


\newcommand{\bugreport}[9]{
\subsection{#1}
	\begin{table}[H]
		\rowcolors{2}{gray!25}{white}
		\centering
		\begin{tabularx}{\textwidth}{| l | X |}
			\hline
			Komponente   & #2       \\
			\hline
			Priorität   & #3       \\
			\hline
			Beschreibung   & #4  \\ 
			\hline
			Konsequenzen   & #5  \\ 
			\hline
			Workaround & #6 \\
			\hline
			Reproduzieren & #7	\\
			\hline
			Reporter  & #8 \\
			\hline
			Feedback Cisco & #9 \\
			\hline
		\end{tabularx}
		\caption{Bug: #1}
	\end{table}
}

\newcommand{\featureRequest}[5]{
	\subsection{#1}
	\begin{table}[H]
		\rowcolors{2}{gray!25}{white}
		\centering
		\begin{tabularx}{\textwidth}{| l | X |}
			\hline
			Komponente   & #2       \\
			\hline
			Beschreibung   & #3  \\ 
			\hline
			Reporter  & #4 \\
			\hline
			Feedback Cisco & #5 \\
			\hline
		\end{tabularx}
		\caption{Feature Request: #1}
	\end{table}
}

\featureRequest
% Titel
{Template Zuweisung}
% Komponente
{\textit{Design $\rightarrow$ Network Profiles}}
% Beschreibung
{Templates können nur Gerätetypen (z.Bsp. Switches oder Router) zugewiesen werden. Es wäre wünschenswert, wenn diese auch verschiedenen Rollen zugewiesen werden könnten. Beispielsweise allen Border Nodes.
}
% Reporter
{Sandro Kaspar}
% Feedback Cisco
{}

\featureRequest
% Titel
{Template Versionierung}
% Komponente
{\textit{Template Editor $\rightarrow$ Create Template}}
% Beschreibung
{Templates können versioniert werden. Allerdings lassen sich ältere Versionen nur anschauen. Folgende Funktionen wären hilfreich:
\begin{itemize}
	\item Restore alter Templateversionen
	\item Diffs zwischen verschiedenen Versionen anzeigen
\end{itemize}
}
% Reporter
{Sandro Kaspar}
% Feedback Cisco
{}

\bugreport
% Titel
{Upload Image}
% Komponente
{\textit{Design $\rightarrow$ Image Repository $\rightarrow$ Import}}
% Dringlichkeit
{Mittel}
% Beschreibung
{Nach dem Uploaden eines Images, wird dieses nicht im WebGui angezeigt. Erst nach mehreren Minuten erscheint das zuvor hochgeladene Image.
}
% Konsequenzen
{Der User geht davon aus, dass der Upload fehlgeschlagen ist und versucht dies erneut.}
% Workaround
{Keiner}
%Reproduzieren
{
	\begin{enumerate}
		\item \textit{Design $\rightarrow$ Image Repository $\rightarrow$ Import}
		\item File auswählen
		\item Warten bis der Upload abgeschlossen ist
		\item Image wird nicht angezeigt
		\subitem Nach mehreren Minuten erscheint das Image
	\end{enumerate}
}
% Reporter
{Sandro Kaspar}
% Feedback Cisco
{}


\bugreport
% Titel
{Claim Device}
% Komponente
{\textit{Provision $\rightarrow$ Inventory}}
% Dringlichkeit
{Niedrig}
% Beschreibung
{Nach dem Hinzufügen oder Claimen eines Devices werden verschiedene Informationen als "null" angezeigt. Hier sollten die korrekten Informationen oder falls noch nicht verfügbar einfach nichts angezeigt werden.
}
% Konsequenzen
{Unschönes UI}
% Workaround
{Keiner}
%Reproduzieren
{
	\begin{enumerate}
		\item \textit{Provision $\rightarrow$ Inventory $\rightarrow$ Unclaimed Devices $\rightarrow$ Claim Device}
	\end{enumerate}
}
% Reporter
{Sandro Kaspar}
% Feedback Cisco
{}

\bugreport
% Titel
{Image Checksum}
% Komponente
{\textit{Design $\rightarrow$ Image Repository}}
% Dringlichkeit
{Mittel}
% Beschreibung
{Wird auf ein Image geklickt, sodass die Detailinformationen angezeigt werden, wird die Checksumme vom "Make a Wish" Button überdeckt. Dies tritt bei FullHD Auflösung oder kleiner auf.
}
% Konsequenzen
{Checksumme nicht lesbar}
% Workaround
{Grössere Auflösung}
%Reproduzieren
{
	\begin{enumerate}
		\item \textit{Design $\rightarrow$ Image Repository $\rightarrow$ Auf Image klicken}
	\end{enumerate}
}
% Reporter
{Sandro Kaspar}
% Feedback Cisco
{}


\bugreport
% Titel
{Provision Status "Failed"}
% Komponente
{\textit{ Provision $\rightarrow$ Devices $\rightarrow$ Inventory}}
% Dringlichkeit
{Niedrig}
% Beschreibung
{Der Provision Status wird als "Failed" angezeigt, auch wenn die Provisionierung funktioniert hat. Dies wird zur Zeit auf ewig so angezeigt, sollte im Verlauf irgendwann einmal etwas fehlgeschlagen sein. Dazu kommt, dass wenn man auf See Details klickt, die Einträge mit Success angezeigt werden. Zuerst muss also herausgefunden werden, bei welchem Workflow dies aufgetreten ist. Mit erneutem Klick auf See Details auf dem Workflow, werden die einzelnen Schritte des Workflows angezeigt.
}
% Konsequenzen
{Verwirrung das Problem vorhanden, obwohl keines besteht}
% Workaround
{Keiner}
%Reproduzieren
{
	
}
% Reporter
{Jessica Kalberer}
% Feedback Cisco
{}


\bugreport
% Titel
{Provision Template Status Failed}
% Komponente
{\textit{ Provision $\rightarrow$ Devices $\rightarrow$ Inventory}}
% Dringlichkeit
{Niedrig}
% Beschreibung
{Der Status des Schrittes im Workflow des Provisioning wird als Failed angezeigt, weil der Name schon gesetzt wurde. Es wird die Meldung "Template IOS Banner Template:2 is already deployed with same params,. Not deploying it." angezeigt. Der Name wurde zwar schon einmal mit dem Template deployed, jedoch handelte es sich nicht um den gleichen.
}
% Konsequenzen
{Verwirrung das Problem vorhanden, obwohl keines besteht}
% Workaround
{Keiner}
%Reproduzieren
{
	\begin{enumerate}
		\item \textit{Provision $\rightarrow$ Devices $\rightarrow$ Inventory}
		\item Gewünschtes Device auswählen
		\item \textit{Actions $\rightarrow$ Provision}
		\item Template für Namensänderung wählen
		\item \textit{Provision}
	\end{enumerate}
}
% Reporter
{Jessica Kalberer}
% Feedback Cisco
{}

\bugreport
% Titel
{Fabric Custom View}
% Komponente
{\textit{ Provision $\rightarrow$ Fabric $\rightarrow$ Layout}}
% Dringlichkeit
{Mittel}
% Beschreibung
{Wenn eine Custom View erstellt wird und diese ausgewählt wird, werden die Devices nicht mehr als Teil der Fabric angezeigt. 
}
% Konsequenzen
{Fabric lässt sich in der Custom View nicht bearbeiten}
% Workaround
{Keiner}
%Reproduzieren
{
	\begin{enumerate}
		\item \textit{Provision $\rightarrow$ Fabric $\rightarrow$ Layout}
		\item Custom View erstellen
		\item Custom View anzeigen
	\end{enumerate}
}
% Reporter
{Sandro Kaspar}
% Feedback Cisco
{}

\bugreport
% Titel
{Fabric Default View}
% Komponente
{\textit{ Provision $\rightarrow$ Fabric $\rightarrow$ Layout}}
% Dringlichkeit
{Niedrig}
% Beschreibung
{Wenn eine Custom View erstellt wird und anschliessend als Default View definiert wird, hat dies keinen Einfluss auf die Default View. Es wird weiterhin die Default View von Cisco angezeigt
}
% Konsequenzen
{Custom Views sind nutzlos, wenn jedes Mal die gewünschte View gewählt werden muss.}
% Workaround
{Keiner}
%Reproduzieren
{
	\begin{enumerate}
		\item \textit{Provision $\rightarrow$ Fabric $\rightarrow$ Layout}
		\item Custom View erstellen
		\item Custom View als Default setzen
		\item Reload
	\end{enumerate}
}
% Reporter
{Sandro Kaspar}
% Feedback Cisco
{}