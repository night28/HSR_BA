\section{Schlussfolgerungen}

\subsection{Erreichte Ziele}
Wie in den Ergebnissen erwähnt, konnten die gesetzten Ziele der Bachelorarbeit grösstenteils erreicht werden.   

 
\subsection{Mögliche Verbesserungen}
Obwohl das DNA Center seit der Studienarbeit stark verbessert wurde, sind immer noch sehr viele Bugs vorhanden. Hier kann durch Fixes dieser Bugs eine grosse Verbesserung erreicht werden. Dasselbe kann für die ISE gesagt werden. Ebenfalls können einzelne Workflows im DNA Center optimiert oder durch Wizards erleichtert werden. Dies würde die Komplexität des Systems reduzieren. Bezüglich der APIs ist es wünschenswert, dass mehr API Endpoints zur Verfügung gestellt und dokumentiert werden.

 \subsection{Zukunft}
Seit vielen Jahren betreiben viele Unternehmen ihre Netzwerke gleich, manuell mit Konsolenkabeln, Telnet, SHH, CLI oder anderen Tools. In der Zeit der Digitalisierung wird es immer schwieriger solch grosse Netze kosteneffizient zu warten. Vor fast zwei Jahren hat sich Cisco zum Ziel gesetzt das Campus Netzwerk neu zu erfinden und dabei das DNA Center entwickelt. Im Frühling 2018 hat Cisco die DNA Assurance eingeführt, um das Monitoring des gesamten Netzwerkes enorm zu vereinfachen. Im Sommer 2018 hat Cisco nun das DNA Center für Entwickler zugänglich gemacht. Dank der neuen offenen DNA Center Platform können über die APIs eigene Anwendungen entwickelt werden.\\

Das DNA Center wurde innerhalb der letzten Jahren stets weiterentwickelt und viele Bugs und Probleme wurden behoben. Es ist zu hoffen, dass dies so weitergeführt wird. \\

Nicht nur Cisco, sondern auch andere Hersteller fördern die Automatisierung des Netzwerkes, um Campus Netzwerke bereitzustellen. Dabei setzen sie ebenfalls auf eine zentrale Plattform, mit welcher das Monitoring und Provisioning ausserordentlich vereinfacht wird. Dies zeigt, dass die Zukunft auch für Campus Netze im Bereich SDN liegt.
