\section{Schlussfolgerungen}

\subsection{Erreichte Ziele}
Wie in den Ergebnissen erwähnt, konnten die gesetzten Ziele der Bachelorarbeit fast vollständig erreicht werden. 
 
 
\subsection{Mögliche Verbesserungen}
Die in dieser Arbeit bisher aufgetretenen Bugs können sicherlich Verbessert werden. Auch wäre es schön, wenn das DNA Center eine Wegweisung, durch zum Beispiel einen Wizard, durch die komplexen Abläufe bereitstellen würde. Dies kann jedoch nun auch durch Verwendung der API und entwicklung eines eigenen Orchestrierungstool etwas umgangen und für sich vereinfacht werden.


Das eigene Orchestrierungstool kann aber auf jeden Fall noch endlos erweitert werden. Dabei können nicht nur verschiedene Abläufe durch einen Wizard vereinfacht werden, welche zur Zeit im DNA Center sehr viel manuellen Aufwand benötigen, sondern auch weitere Eigenentwicklungen implementiert werden.


 \subsection{Zukunft}
Seit vielen Jahren betreiben viele Unternehmen ihre Netzwerke gleich, manuell mit Konsolenkabeln, Telnet, SHH, CLI oder anderen Tools. In der Zeit der Digitalisierung wird es immer schwieriger solch grosse Netze kosteneffizient zu Warten. Vor fast zwei Jahren hat sich Cisco zum Ziel gesetzt das Netzwerk neu zu erfinden und dabei das DNA Center entwickelt. Im Frühling 2018 hat Cisco das DNA Assurance eingeführt, um das Monitoring des gesamten Netzwerkes enorm zu vereinfachen. Im Sommer 2018 hat Cisco nun das DNA Center für Entwickler zugänglich gemacht. Dank der neuen offenen DNA Center Platform können nun über die APIs eigene Anwendungen entwickelt werden.\\

Cisco hat das DNA Center innerhalb der letzten Jahren immer weiterentwickelt und viele vorher vorhandene Bugs behoben. Bleibt nur zu hoffen, dass sich dies so weiterentwickelt. \\

Doch nicht nur Cisco, sondern auch andere Konkurrenten fördern die Automatisierung des Netzwerkes, um Unternehmensweite Netzwerke bereitzustellen. Dabei setzen sie ebenfalls auf eine zentrale Plattform, mit welcher das Monitoring und Provisioning ausserordentlich vereinfacht wird.
