\section{Management Summary}

\subsection{Ausgangslage}
Diese Bachelorarbeit ist eine Folgearbeit der Studienarbeit Software-Defined Netzwerk im Campus Bereich. Ziel der Studienarbeit war die Evaluation des Cisco Digital Network Architecture (DNA) Center, der Software-Defined Network (SDN) Lösung von Cisco, für die Führungsunterstützungsbasis (FUB) der Schweizer Armee. In der Studienarbeit wurden mit den vorhandenen Geräten zwei Standorte Jona und Rapperswil geplant. Auf Grund der aufgetretenen Probleme und Bugs, wurde entschieden sich nur auf die Seite Rapperswil mit mehreren Devices und zwei Gebäuden zu konzentrieren. Hier konnte eine funktionierende Fabric mit zwei Sites erstellt und verschiedene Policies implementiert werden. Somit konnten die vorgegebenen Use Cases, wie zum Beispiel die Definition von Benutzerprofilen, Benutzermobilität oder Reporting der Netzwerkaktivitäten umgesetzt und getestet werden.

\subsection{Vorgehen und Technologien}
Nun soll in dieser Bachelorarbeit in einem ersten Schritt für die SDN Lösung analysiert und sämtliche kritische Elemente identifiziert werden. In einem zweiten Schritt werden die kritischen Elemente abgesichert und eine Empfehlung für das Deployment eines krisensicheren SDN erarbeitet. Zum Abschluss erfolgt das Abstrahieren einzelner Elemente, welche im Verlauf der Bachelorarbeit noch in Use Cases definiert werden, aus dem DNA Center in ein eigenes Web Interface. 

\subsection{Ergebnisse}
Als Ergebnis dieser Bachelorarbeit wird ein möglichst krisensicherer Prototyp zur Verfügung stehen. Der Prototyp besteht aus den Cisco Komponenten, sowie Eigenentwicklungen, die zusätzliche Features implementieren. Zudem steht eine Dokumentation des Systems zur Verfügung, welche eine Empfehlung für das Deployment eines krisensicheren SDN beinhaltet.

\subsection{Ausblick}
Die Resultate dieser Arbeit können dazu dienen, ein möglichst krisensichere SDA Lösung in einer produktiven Umgebung in Betrieb zu nehmen. Zudem kann der Prototyp um zusätzliche Funktionen erweitert, an bestehende oder neue Systeme angebunden oder mit alternativen Lösungen verglichen werden. Über das eigen Erstellte Web Interface können ausserdem einzelne Standard-Aufgaben unabhängig vom DNA Center ausgeführt werden.
 
