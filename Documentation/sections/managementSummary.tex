\section{Management Summary}

\subsection{Ausgangslage}
Diese Bachelorarbeit ist eine Folgearbeit der Studienarbeit \textit{Software-Defined Netzwerk im Campus Bereich} \cite{studienarbeit}. Ziel der Studienarbeit war die Evaluation des Cisco Digital Network Architecture (DNA) Center, der Software-Defined Network (SDN) Lösung von Cisco, für die Führungsunterstützungsbasis (FUB) der Schweizer Armee. In der Studienarbeit wurden mit den vorhandenen Geräten zwei Standorte geplant. Auf Grund von aufgetretenen Problemen und Bugs wurde entschieden, sich nur auf einen Standort mit mehreren Devices und zwei Gebäuden zu konzentrieren. Für diesen Standort wurde eine Fabric mit zwei Sites erstellt und diverse Policies implementiert. Somit konnten die vorgegebenen Use Cases, wie zum Beispiel die Definition von Benutzerprofilen, Benutzermobilität oder Reporting der Netzwerkaktivitäten umgesetzt und getestet werden.

\subsection{Vorgehen}
Nun soll in dieser Bachelorarbeit in einem ersten Schritt die SDN Lösung analysiert und sämtliche kritischen Elemente identifiziert werden. Zudem wurden Ideen entwickelt, wie diese kritischen Komponenten abgesichert und redundant betrieben werden können, sodass der Betrieb auch bei Ausfall von Komponenten oder Verbindungen jederzeit ge\-währ\-leis\-tet bleibt. In einem zweiten Schritt werden die kritischen Elemente wenn möglich abgesichert und eine Empfehlung für das Deployment eines krisensicheren SDN erarbeitet. Neu hinzu kam auch die Virtualisierungsplattform ENCS 5400, welche verwendet wurde, um einen autonomen Standort zu simulieren. Zum Abschluss erfolgt das Abstrahieren einzelner Elemente, welche im Verlauf der Bachelorarbeit noch in Use Cases definiert wurden, aus dem DNA Center in ein eigen entwickeltes Orchestrierungstool. 

\subsection{Ergebnisse}
Als Ergebnis dieser Bachelorarbeit steht ein möglichst krisensicherer Prototyp zur Ver\-fü\-gung. Der Prototyp besteht aus den Cisco Komponenten sowie Eigenentwicklungen, die zu\-sätz\-liche Features implementieren. Zudem steht eine Dokumentation des Systems zur Verfügung, welche eine Analyse der kritischen Komponenten, eine Empfehlung für das Deployment eines krisensicheren SDN, sowie eine Dokumentation über das entwickelte Orchestrierungstool beinhaltet. Über dieses Tool können ausserdem einzelne Standard-Aufgaben unabhängig vom DNA Center ausgeführt werden. Dies soll zum einen Aufgaben vereinfachen, in dem diese beispielsweise mit einem Wizard durchlaufen werden können. 
 
\subsection{Ausblick}
Die Resultate dieser Arbeit können dazu dienen, eine möglichst krisensichere SDA Lösung in einer produktiven Umgebung in Betrieb zu nehmen. Zudem kann der Prototyp um zusätzliche Funktionen erweitert, an bestehende oder neue Systeme angebunden oder mit alternativen Lösungen verglichen werden. Durch das entwickelte Orchestrierungstool wurden die Möglichkeiten der DNA Center APIs aufgezeigt. Diese bieten schon viele Möglichkeiten. Allerdings sind viele API Endpoint noch nicht, oder nicht vollständig, dokumentiert. Dies soll in zukünftigen Releases aber verbessert werden.
