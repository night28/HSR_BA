\section{Persönliche Summaries}

\subsection{Sandro Kaspar}

Ich war sehr erfreut, dass wir eine Folgearbeit für unsere Studienarbeit machen durften. Insbesondere weil ich mich schon immer sehr stark für Netzwerktechnologien interessiert habe. Mit dieser Arbeit wurde mir zudem die Möglichkeit geboten, mit sehr aktuellen Technologien und Produkten zu arbeiten, die ich aus dem Arbeitsalltag noch nicht kenne. Des Weiteren ist die Aufgabenstellung auf Grund der Grösse und Struktur des Industriepartners sehr herausfordernd.\\
Da das DNA Center in der Studienarbeit noch sehr neu war, freute ich mich darauf, zu sehen wie sich dieses weiterentwickelt hat, was verbessert wurde und welche neuen Funktionen hinzugekommen sind. Ebenfalls sehr spannend war der Einsatz des ENCS 5400, ein Produkt, das ich in dieser Form zuvor noch nicht kannte. 
Erfreulicherweise kann ich sagen, dass sich das DNA Center sehr stark verbessert hat. Die Softwarequalität hat stark zugenommen und es wurden neue Features hinzugefügt, welche die Arbeit mit dem DNA Center erleichtern. An den eingesetzten Produkten, dem DNA Center, der ISE und dem ENCS 5400 gibt es sicherlich immer noch viel Verbesserungspotenzial, besoners was die Softwarequalität und die intuitive Bedienung der Produkte anbelangt. Wenn diese aber wie bisher stetig weiterentwickelt werden, bin ich zuversichtlich, dass sich diese Technologien durchsetzen werden.\\
Die Arbeit im Team funktionierte meiner Meinung nach sehr gut. Ebenfalls sehr gut gestaltete sich die Zusammenarbeit mit unseren Betreuern, Experten und dem Industriepartner. \\
Abschliessend kann ich sagen, dass ich während dieser Arbeit sehr viel gelernt habe und meine Fähigkeiten weiterentwickeln konnte. Es war spannend mit modernsten Produkten und zukunftsweisenden Technologien zu arbeiten und zu sehen wie sich diese bewähren.

\subsection{Jessica Kalberer}

Schon die Studienarbeit \textit{Software-Defined Netzwerk im Campus Bereich} hat mich seit Anfang sehr interessiert. Mich in das Netzwerk der Zukunft, dem Software Defined Network, einarbeiten zu können, hat mich sehr gefreut und ich bin dankbar für diese Chance mit einer so guten Lab Hardware arbeiten zu dürfen. Umso mehr hat es mich natürlich gefreut, dass wir als Folgearbeit diese Bachelorarbeit \textit{Krisenresistente Software Defined Netzwerke} machen durften. \\
Da in der Studienarbeit mit den relativ neuen DNA Center diverse Probleme auftraten, wollte ich unbedingt wissen, ob dies nun in den neuen Releases anders wird. Ich war wirklich positiv überrascht, dass viele Bugs vor allem in der LAN Automation und dem Provisioning behoben wurden. Jedoch haben wir auch in der Bachelorarbeit wieder feststellen müssen, dass es wiederrum neue Bugs gegeben hat, welche wir grösstenteils auch in diesem Dokument festgehalten haben. Jedoch denke ich, dass wenn die Verbesserung und Weiterentwicklung des DNA Center weiterhin so bleibt, dieses Produkt auch enorm Potential hat. \\
Neu dazu kamen in der Bachelorarbeit die Virtualisierungsplattform ENCS 5400 und die DNA Center Plattform, welche seit September 2018 veröffentlicht wurde. Auf den ersten Blick auf die DNA Center Plattform war ich positiv überrascht, wie viele APIs dokumentiert und teils auch direkt im GUI mit einem Aufruf getestet werden können. Doch der Schein trügt, den bei diesen dokumentierten API Calls handelt es sich um andere, als die die im DNA Center verwendet werden. Leider sind auch nicht Ansatzweise so viele API Calls dokumentiert, wie wir in der Umsetzung des dritten Teiles gebraucht hätten. Trotzdem war dieser Teil sehr spannend und kann meiner Meinung auch endlos erweitert werden. \\
Die Arbeit im Team empfand ich als angenehm. Auch konnte ich vieles von Sandro lernen, da er viel mehr praktische Erfahrung hat. Auch möchte ich mich bei unserem Betreuer, sowie auch Urs, dem Industriepartner und Patrick Mosimann für die unglaublich gute Zusammenarbeit und die vielen wertvollen Inputs bedanken. \\
Zum Schluss kann ich sagen, dass es für mich eine sehr spannende, herausfordernde und enorm lehrreiche Arbeit war.