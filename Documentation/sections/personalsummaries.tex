\section{Persönliche Summaries}

\subsection{Sandro Kaspar}

Ich war sehr erfreut, dass wir eine Folgearbeit für unsere Studienarbeit machen durften. Insbesondere weil ich mich schon immer sehr stark für Netzwerktechnologien interessiert habe. Mit dieser Arbeit wurde mir zudem die Möglichkeit geboten, mit sehr aktuellen Technologien und Produkten zu arbeiten, die ich aus dem Arbeitsalltag noch nicht kenne. Des Weiteren ist die Aufgabenstellung auf Grund der Grösse und Struktur des Industriepartners sehr herausfordernd.\\
Da das DNA Center in der Studienarbeit noch sehr neu war, freute ich mich darauf, zu sehen wie sich dieses weiterentwickelt hat, was verbessert wurde und welche neuen Funktionen hinzugekommen sind. Ebenfalls sehr spannend war der Einsatz des ENCS 5400, ein Produkt, das ich in dieser Form zuvor noch nicht kannte. 
Erfreulicherweise kann ich sagen, dass sich das DNA Center sehr stark verbessert hat. Die Softwarequalität hat stark zugenommen und es wurden neue Features hinzugefügt, welche die Arbeit mit dem DNA Center erleichtern. An den eingesetzten Produkten, dem DNA Center, der ISE und dem ENCS 5400 gibt es sicherlich immer noch viel Verbesserungspotenzial, besoners was die Softwarequalität und die intuitive Bedienung der Produkte anbelangt. Wenn diese aber wie bisher stetig weiterentwickelt werden, bin ich zuversichtlich, dass sich diese Technologien durchsetzen werden.\\
Die Arbeit im Team funktionierte meiner Meinung nach sehr gut. Ebenfalls sehr gut gestaltete sich die Zusammenarbeit mit unseren Betreuern, Experten und dem Industriepartner. \\
Abschliessend kann ich sagen, dass ich während dieser Arbeit sehr viel gelernt habe und meine Fähigkeiten weiterentwickeln konnte. Es war spannend mit modernsten Produkten und zukunftsweisenden Technologien zu arbeiten und zu sehen wie sich diese bewähren.

\subsection{Jessica Kalberer}
